%%%%%%%%%%%%%%%%%%%%%%%%%%%%%%%%%%%%%%%%%
% Freeman Curriculum Vitae
% XeLaTeX Template
% Version 2.0 (19/3/2018)
%
% This template originates from:
% http://www.LaTeXTemplates.com
%
% Authors:
% Vel (vel@LaTeXTemplates.com)
% Alessandro Plasmati
%
% License:
% CC BY-NC-SA 3.0 (http://creativecommons.org/licenses/by-nc-sa/3.0/)
%


% NOTICE: This template must be compiled with XeLaTeX, the line above should
% ensure this happens automatically but if it doesn't you will need to specify 
% XeLaTeX as the engine in your editor or script
% 
%%%%%%%%%%%%%%%%%%%%%%%%%%%%%%%%%%%%%%%%%

%----------------------------------------------------------------------------------------
%	PACKAGES AND OTHER DOCUMENT CONFIGURATIONS
%----------------------------------------------------------------------------------------
% !TEX program = xelatex

\documentclass[10pt]{article} % Font size, can be: 10pt, 11pt or 12pt

%%%%%%%%%%%%%%%%%%%%%%%%%%%%%%%%%%%%%%%%%
% Freeman Curriculum Vitae
% Structure Specification File
% Version 1.0 (19/3/2018)
%
% This template originates from:
% http://www.LaTeXTemplates.com
%
% Authors:
% Vel (vel@LaTeXTemplates.com)
% Alessandro Plasmati
%
% License:
% CC BY-NC-SA 3.0 (http://creativecommons.org/licenses/by-nc-sa/3.0/)
% 
%%%%%%%%%%%%%%%%%%%%%%%%%%%%%%%%%%%%%%%%%

%----------------------------------------------------------------------------------------
%	PACKAGES AND OTHER DOCUMENT CONFIGURATIONS
%----------------------------------------------------------------------------------------

\usepackage{etoolbox} %Required for conditional statements
\usepackage{amsmath}
\setlength\parindent{0pt} % Stop paragraph indentation
% To get the total page numbers (\pageref{LastPage})
\usepackage{lastpage}
\usepackage{supertabular} % Required for the supertabular environment which allows tables to span multiple pages so sections with tables correctly wrap across pages

%----------------------------------------------------------------------------------------
%	DOCUMENT MARGINS
%----------------------------------------------------------------------------------------

\usepackage{geometry} % Required for adjusting page dimensions and margins

\geometry{%
  margin=15mm,
  headsep=0mm,
  headheight=0mm,
  footskip=5mm,
  includehead=true,
  includefoot=true
}

%----------------------------------------------------------------------------------------
%	COLUMN LAYOUT
%----------------------------------------------------------------------------------------
\usepackage{hyperref}
\usepackage{paracol} % Required for creating multi-column layouts that can span pages automatically

\columnratio{0.55,0.45} % The relative ratios of the two columns in the CV

\setlength\columnsep{0.05\textwidth} % Specify the amount of space between the columns

%----------------------------------------------------------------------------------------
%	FONTS
%----------------------------------------------------------------------------------------

\usepackage{fontspec} % Required for specifying custom fonts under XeLaTeX



\newfontfamily\cvtextfont[Path=fonts/]{freebooterscript} % Create a new font family for the cursive font Freebooter Script, provided with the template in the fonts folder




\newfontfamily{\FA}[Path=fonts/]{FontAwesome} % Create a new font family for FontAwesome icons, provided with the template in the fonts folder
\input{fonts/fontawesomesymbols-xeluatex.tex} % Load a file to create shortcuts to the icons, see icon examples and their shortcuts in fontawesome.pdf in the fonts folder

% \usepackage[sf,scale=0.95]{libertine} % Load Libertine as a \sffamily font for sans serif titles

% \usepackage[sf,scale=0.95]{libertine} % Load Libertine as a \sffamily font for sans serif titles

%----------------------------------------------------------------------------------------
%	COLOURS AND LINKS
%----------------------------------------------------------------------------------------

\usepackage[usenames,svgnames]{xcolor} % Allows the definition and use of custom colours

\definecolor{text}{HTML}{2b2b2b} % Main document font colour, off-black
\definecolor{headings}{HTML}{002fa7} % Dark lavender colour for headings
\definecolor{shade}{HTML}{F0F8FF} % Pinky colour for the contact information box
\definecolor{linkcolor}{HTML}{191970} % 25% desaturated headings colour for links
% Other colour options: shade=B9D7D9 and linkcolor=A40000; shade=D4D7FE and linkcolor=FF0080

% For preset colours that can be used by their names without having to define them, see: https://www.latextemplates.com/svgnames-colors

\color{text} % Set the default text colour for the whole document to the colour defined as 'text' above


% Control PDF metadata and links
\newcommand{\Title}{Curriculum Vit\ae\ Summary}
\newcommand{\FirstName}{Chris "Le"}
\newcommand{\LastName}{Wang}
\newcommand{\Initials}{C}
\newcommand{\MyName}{\FirstName\ \LastName}
\newcommand{\Me}{\textbf{\LastName, \Initials}}  % For citations

\hypersetup{
  colorlinks,
  allcolors=mediumblue,
  breaklinks=true,
  pdftitle={\Title{} - \MyName},
  pdfauthor={\MyName},
}

%------------------------------------------------

\usepackage{hyperref} % Required for links

\hypersetup{colorlinks, breaklinks, urlcolor=linkcolor, linkcolor=linkcolor} % Set up links and their colours


%----------------------------------------------------------------------------------------
%	HEADERS & FOOTERS
%----------------------------------------------------------------------------------------

\usepackage{fancyhdr} % Required for customising headers and footers

\pagestyle{fancy} % Enable custom headers and footers

\fancyhf{} % This suppresses all headers and footers by default, add headers and footers in the template file as per the example

\renewcommand{\headrulewidth}{0pt} % Remove the default rule under the header

%----------------------------------------------------------------------------------------
%	SECTIONS
%----------------------------------------------------------------------------------------

\usepackage[nobottomtitles*]{titlesec} % Required for modifying sections, the nobottomtitles* is required for pushing section titles to the next page when they are close to the bottom of the page

\renewcommand{\bottomtitlespace}{0.1\textheight} % Modify the minimal space required from the bottom margin not to move the title to the next page

\titleformat{\section}{\color{headings}\scshape\LARGE\raggedright}{}{0em}{}[\color{black}\titlerule] % Define the \section format

\titlespacing{\section}{0pt}{0pt}{5pt} % Spacing around section titles, the order is: left, before and after

%----------------------------------------------------------------------------------------
%	CUSTOM COMMANDS
%----------------------------------------------------------------------------------------
% literature links--use doi if you can
\newcommand{\doi}[2]{\emph{\addfontfeature{Color=linkblue}\href{http://dx.doi.org/#1}{{#2}}}}
\newcommand{\link}[2]{\emph{\addfontfeature{Color=linkblue}\href{#1}{{#2}}}}
\newcommand{\ads}[2]{{\addfontfeature{Color=linkblue}\href{http://adsabs.harvard.edu/abs/#1}}{{#2}}}
\newcommand{\arxiv}[2]{\emph{\addfontfeature{Color=linkblue}\href{http://arxiv.org/abs/#1}{{#2}}}}
\newcommand{\isbn}[1]{{\footnotesize(\textsc{isbn:}{#1})}}

% Journal names.
\newcommand{\aj}{AJ}
\newcommand{\apj}{ApJ}
\newcommand{\pasp}{PASP}
\newcommand{\mnras}{MNRAS}

\newcommand{\pubslist}{%
    \rightmargin=0in
    \leftmargin=0.15in
    \topsep=0ex
    \partopsep=0pt
    \itemsep=1.25ex
    \parsep=0pt
    \itemindent=-1.0\leftmargin
    \listparindent=0.0\leftmargin
    \settowidth{\labelsep}{~}
}


% Command for entering a separate qualification
\newcommand{\workposition}[4]{
	\textsc{#1} & \expandafter\ifstrequal\expandafter{#2}{}{}{{\raggedright\large{\textbf{#2}}}\\[4pt]} % Duration and Institution
	%\expandafter\ifstrequal\expandafter{#4}{}{}{& {}\\} % Honours, achievements or distinctions (e.g. first class honours)
	\expandafter\ifstrequal\expandafter{#3}{}{}{& #3\\} % Advisors
	\expandafter\ifstrequal\expandafter{#4}{}{}{& #4\\[6pt]} % description
}

% Command for entering a separate qualification
\newcommand{\educationentry}[5]{
	\textsc{#1} & #2\\ % Duration and school
        \expandafter\ifstrequal\expandafter{#4}{}{}{& #3\\} %Majors
	%\expandafter\ifstrequal\expandafter{#4}{}{}{& {}\\} % Honours, achievements or distinctions (e.g. first class honours)
	\expandafter\ifstrequal\expandafter{#4}{}{}{& #4\\} % GPA
	\expandafter\ifstrequal\expandafter{#5}{}{}{& #5\\[6pt]} % Minor
}

% Command for entering a separate table row -- used as a generic visual element for any section that uses a two column table
\usepackage{amsmath}
\newcommand{\tableentry}[3]{
\vspace*{-1mm}
	{\raggedleft\textsc{#1{\hspace{6pt}}}\par}
	\expandafter\ifstrequal\expandafter{\textbf{#2}{}}{}{}
	{{\raggedright \textbf{#2}{}}\\}
	%\expandafter\ifstrequal\expandafter{#3}{}{\\}{\\[6pt]}
	% First the heading, then content, then a conditional insertion of whitespace if the third parameter has any content in it
	\expandafter\ifstrequal\expandafter{#3}{}{}{& #3\\}
\vspace*{-5mm}
}

\newcommand{\awardsentry}[3]{
%\vspace*{-1mm}
	\textsc{#1} &
	\expandafter\ifstrequal\expandafter{#2}{}{}{{\raggedright\textbf{#2}}\\[2pt]} % Award title
	\expandafter\ifstrequal\expandafter{#3}{}{}{& #3\\} % Description

}

\newcommand{\talkentry}[3]{
	\textsc{#1} &
	\expandafter\ifstrequal\expandafter{#2}{}{}{{\raggedright\textbf{#2}}\\[2pt]} % Time Title
 \expandafter\ifstrequal\expandafter{#3}{}{}{& {\textit{#3}}\\} %conference
}


\newcommand{\teaching}[2]{
	\textsc{#1} &
	\expandafter\ifstrequal\expandafter{#2}{}{}{{{#2}}\\[2pt]} % Award title
}



\newcommand{\talks}[2]{
	{#1} \hfill {#2} \medskip
}

% Command for entering a long-form description where there is a title on one line and a paragraph description below it
\newcommand{\longformdescription}[2]{
	\textit{#1}\\[3pt]
	#2\medskip
}

% Command for entering a publication in long-form format
\newcommand{\longformpublication}[1]{
	#1\medskip
}

% Command for entering a publication as a short DOI (digital object identifier) string to the publication, a link is automatically created
\newcommand{\doipublication}[4]{
	#1 & % Year
	\href{http://dx.doi.org/#2}{\expandafter\ifstrequal\expandafter{#3}{firstauthor}{\textbf{doi:#2}}{doi:#2}}% DOI string and if "firstauthor" is entered for the 3rd argument, this makes the DOI string bold indicating a first author publication
	\expandafter\ifstrequal\expandafter{#4}{}{\\}{\\[3pt]} % Conditional insertion of whitespace if the 4th parameter has any content in it
} % Include the file that specifies the document structure

% Headers and footers can be added with the \lhead{} \rhead{} \lfoot{} \rfoot{} commands
% Example right footer:
%\rfoot{\color{headings}{\sffamily Last update: \today. Typeset with Xe\LaTeX}}

%----------------------------------------------------------------------------------------

\usepackage{array}
\usepackage{etaremune}
\newcommand{\PreserveBackslash}[1]{\let\temp=\\#1\let\\=\temp}
\newcolumntype{C}[1]{>{\PreserveBackslash\centering}p{#1}}
\newcolumntype{R}[1]{>{\PreserveBackslash\raggedleft}p{#1}}
\newcolumntype{L}[1]{>{\PreserveBackslash\raggedright}p{#1}}
\usepackage{longtable}


% Configure a fancy footer
\newcommand{\Separator}{\hspace{3pt}|\hspace{3pt}}
\newcommand{\FooterFont}{\footnotesize\color{black}}
\pagestyle{fancy}
\fancyhf{}
\lfoot{%
  \FooterFont{}
  \MyName{}
  \Separator{}
  \Title{}
}
\rfoot{%
  \FooterFont{}
  Last updated: \today{}
  \Separator{}
  \thepage\space of\space \pageref*{LastPage}
}
\renewcommand{\headrulewidth}{0pt}
\renewcommand{\footrulewidth}{1pt}
\preto{\footrule}{\color{lightgray}}


\begin{document}


 % Begin the multi-column environment
\begin{paracol}{2}
%----------------------------------------------------------------------------------------
%	NAME AND CURRICULUM VITAE TEXT
%----------------------------------------------------------------------------------------

\parbox[top][0.12\textheight][c]{\linewidth}{ % Parbox to hold the author name and CV text; fixed height to match the coloured box to the right, centred vertically and full line width
	\vspace{-0.04\textheight} % Reduce whitespace above the parbox to separate it from the main content
	\centering % Centre text
	{\sffamily\Huge Chris "Le" Wang}\\\medskip % Your name
	{\Huge\color{headings}\cvtextfont Curriculum Vitae}
}

\switchcolumn % Switch to the next paracol column
%----------------------------------------------------------------------------------------
%	COLOURED CONTACT DETAILS BOX
%----------------------------------------------------------------------------------------

\parbox[top][0.12\textheight][c]{\linewidth}{ % Parbox to hold the colour box; fixed height to match the name/CV text to the left, centred vertically and full line width
	\vspace{-0.04\textheight} % Reduce whitespace above the parbox to separate it from the main content
	\colorbox{shade}{ % Create the coloured box
		\begin{supertabular}{p{0.05\linewidth}|p{0.775\linewidth}} % Start a table with two columns, the table will ensure everything is aligned
			\raisebox{-1pt}{\faHome} & 110 W 39th St, Baltimore, MD, USA \\ % Address
			\raisebox{-1pt}{\faPhone} & +1 (443) 254-2113 \\ % Phone number
			\raisebox{0pt}{\small\faEnvelope} & \href{mailto:lwang178@jhu.edu}{lwang178@jhu.edu} \\ % Email address
			\raisebox{-1pt}{\small\faDesktop} & \href{https://astrochriswang.com}{astrochriswang.com} \\ % Website
			\raisebox{-1pt}{\faGithub} & \href{https://github.com/Chrrrrris}{https://github.com/Chrrrrris} \\ % GitHub profile
			\raisebox{-1pt}{\faLinkedinSquare} & \href{https://www.linkedin.com/in/chris-wang-a85524223/}{https://www.linkedin.com/in/chris-wang-a85524223/} \\ % LinkedIn profile
			% See fontawesome.pdf in the fonts folder for all icons you can use
		\end{supertabular}
	}
}

\end{paracol}
%----------------------------------------------------------------------------------------
%	MAJOR RESEARCH PROJECT
%----------------------------------------------------------------------------------------

%\medskip % Extra whitespace before the next section
%----------------------------------------------------------------------------------------
%	EDUCATION
%----------------------------------------------------------------------------------------
\section{Education} 

% Blank \educationentry{} command to add another degree:

%\educationentry{} % Duration
%{} % Degree
%{} % Honours, achievements or other comments
%{} % Department
%{} % Institution

% All 5 parameters must be supplied but any can be empty if you don't need them

%------------------------------------------------
\vspace{-0.4cm}
\begin{longtable}{L{4cm}@{\hskip 0.15in}L{13.5cm}} % Start a table with two columns, the table will ensure everything is aligned
	%------------------------------------------------
	
	
	%------------------------------------------------
	
	\collge{Aug. 2021 -- May 2025} % Duration
	{\textbf{Johns Hopkins University}, \textit{Baltimore, MD, USA}} % Degree
	{BSc in Computer Science, Physics, and Applied Mathematics \& Statistics} % Department
	{GPA: 3.95/4.00} % Major
	{Minor: Mathematics} % Minor
        {Grad courses taken: 9}
	
	%------------------------------------------------
	
	% \educationentry{Aug. 2018 -- May 2021} % Duration
	% {\textbf{Hangzhou Foreign Languages School}, \textit{Hangzhou, China}} % Degree
	% {GCE A-Level \& Chinese High School Diploma}
 %        {GPA: 4.0/4.0}
	%------------------------------------------------

\end{longtable}
%----------------------------------------------------------------------------------------
%	RESEARCH EXPERIENCE
%----------------------------------------------------------------------------------------

\section{Research Projects}

% Blank \workposition command to add another job:

%\workposition{} % Duration
%{} % FT/PT (full time or part time)
%{} % Employer
%{} % Job title, advisors
%{} % Project title

% All 5 parameters must be supplied but any can be empty if you don't need them

%------------------------------------------------

\vspace{-0.4cm}

\begin{longtable}{L{4cm}@{\hskip 0.15in}L{13.5cm}} % Start a table with two columns, the table will ensure everything is aligned

\workposition{Jan. 2022 -- present} % Duration
 % FT/PT (full time or part time)
{\normalsize Schlaufman Exoplanet Group} % Employer
{Advisor: Prof. Kevin C. Schlauman, Dr. Matthew S. Clement} % Job title
{\begin{itemize}
\vspace{-0.4cm}
    \item \textbf{Unresolved Binary Star Rejection}:  Assembled photometry for every star confirmed as an open cluster member by Gaia. Designed algorithms that fit Hertzsprung–Russell diagrams and reject unresolved binary stars.
    \item \textbf{Stellar Elemental Abundance and Planet Formation}: Simulated the mass evolution of stellar surface convective zone using Modules for Experiments in Stellar Astrophysics (MESA). Showed there is no relationship between stellar photospheric elemental abundance pattern and planet formation.
    \item \textbf{Planet Formation with N-body Simulation}: 
    \begin{itemize}
        \item investigate the relative importance between pebble accretion and planetesimal accretion to the outcomes of planet formation directly with numerical simulations and exoplanet demographics with Mercury6.% Sample outcomes from Mercury6 that simulate the stochastic stage of planet formation with given initial conditions.
        \item investigate the stability of mean-motion-resonance chains for TOI-700 system with Mercury6.
        \item investigate the planet formation outcomes for MK-dwarf systems with varying disk mass, which is then combined with a volatile growth model to track planets' atmospheric and mantle composition of H$_2$O, N$_2$, and CO$_2$.
    \end{itemize}
    % \item \textbf{Stability of Mean-motion-resonance Chains}: investigated the stability of resonant chains for TOI-700 system with N-body simulations from Mercury6.
\end{itemize}} % Project
\workposition{May 2022 -- Present} % Duration
 % FT/PT (full time or part time)
{\normalsize Sing Exoplanet Group} % Employer
{Advisors: Prof. David K. Sing, Zafar Rustamkulov} % Job title
{\begin{itemize}
\vspace{-0.4cm}
    \item \textbf{JWST data reduction pipeline development}: Optimized JWST NIRSpec data reduction pipeline using nested sampling to extract transit light curves; reduced the light curve extraction runtime by an order of magnitude. Integrated the capability to reduce JWST NIRISS/SOSS data to the team's JWST data reduction pipeline originally designed for NIRSpec.
    \item \textbf{Transmission Spectroscopy}:  Extracted transmission spectra for WASP-96b, HAT-P-14b, and K2-18b. Combined transmission spectrum from SOSS with that derived from various space-based and ground-based observatories and retrieved atmospheric properties. 
\end{itemize}} % Project
\pagebreak
\workposition{Apr. 2020 -- Nov. 2020} % Duration
 % FT/PT (full time or part time)
{\normalsize Polar Research Institute of China (PRIC)} % Employer
{Advisor: Dr. Peng Jiang} % Job title
{\begin{itemize}
\vspace{-0.4cm}
    \item \textbf{General Relativity Testing}: coauthored a paper (three authors contributed equally) on the possibility to detect general relativity in exoplanet systems. Derived an analytic formula evaluating the sensitivity of perihelion’s precession in radial velocity measurements. Explored the possibility to detect general relativistic precession in exoplanets through radial velocity measurements using RadVel.
\end{itemize}}
 

% \workposition{Current, from Feb 2021} % Duration
% {FT} % FT/PT (full time or part time)
% {Stony Brook University/Flatiron Institute (CCA)} % Employer
% {Advisor: Dr. Phil Armitage} % Job title
% {Migration of Planets and Stars in Protoplanetary and AGN Disks} % Project

% %------------------------------------------------
% \workposition{Jul 2021 -- Aug 2021} % Duration
% {PT} % FT/PT (full time or part time)
% {Eureka Scientific} % Employer
% {Advisor: Dr. Thayne Currie} % Job title
% {Direct Imaging and Astrometry of an Extrasolar Planet} % Project

\workposition{Jan 2020 -- May 2020} % Duration
 % FT/PT (full time or part time)
{\normalsize Duke University} % Employer
{Advisor: Prof. Thomas C. Mehen} % Job title
{\begin{itemize}
\vspace{-0.4cm}
 \item \textbf{Quantum Computing}: coauthored a paper on testing  Bell's and Mermin's inequalities on quantum computers. Designed two-Qbit and three-Qbit quantum circuits and analyzed simulation results.
 \end{itemize}}  % Project

%------------------------------------------------

%------------------------------------------------
\end{longtable}
%------------------------------------------------

\vspace*{1mm} % Standardise the whitespace after this section and before the next (the custom command adds too much otherwise)

%----------------------------------------------------------------------------------------
%	PUBLICATIONS
%----------------------------------------------------------------------------------------

\section{Publications}

% Example \longformdescription{} command to add another publication:

%\longformpublication{Reference (format this manually as desired)}

%------------------------------------------------
\begin{etaremune}
\item \longformpublication{\textbf{Wang, C. L.},  Sing, D. K., \& Rustamkulov, Z.}, {}{"Transmission Spectroscopy of WASP-96b with Combined VLT, Hubble, and NIRISS/SOSS Retrieval"} \textit{in prep.}
\item \longformpublication{Liu, R.*, \textbf{Wang, C. L.}*, Rustamkulov, Z., \&  Sing, D. K.}, {}{"Rereduction and Calibration of JWST NIRSpec and NIRISS Commissioning Data on Hat-p-14b with the Latest Methods"} \textit{in prep.} (*: Co-first author)

\item \longformpublication{\textbf{Wang, C. L.} \& Schlaufman, K. C.}, {}{"Elemental Abundance Trends with Condensation Temperature are Unrelated to Planet Formation"} \textit{in prep.}

\item \longformpublication{Gou, X.*, Pan, X.*, \textbf{Wang, C. L.}*, \href{https://iopscience.iop.org/article/10.1088/1755-1315/658/1/012051}{"General Relativity Testing in Exoplanetary Systems
"} \textit{IOP Conf. Ser.: Earth Environ. Sci.} (2021).} (*: Equal contributions).

\item \longformpublication{Zheng, Y., Wang, X., \textbf{Wang, C. L.} et al.,  \href{https://trebuchet.public.springernature.app/get_content/e19b1eb6-daac-4e48-a946-f55457e622aa}{"Test of Bell's and Mermin's inequalities on Quantum Computer"} \textit{2020 2nd International Conference on Information Technology and Computer Application} (2020).}
\end{etaremune}




\section{Talks \& Presentations}

% Example \tableentry{} command to add another line:

%\tableentry{Heading}{Content}{spaceafter}

% All 3 parameters must be supplied but any can be empty if you don't need them
% A "spaceafter" value in the third parameter will add some vertical space -- this is to be used between headings

%------------------------------------------------

 % Start a table with two columns, the table will ensure everything is aligned
	
	%------------------------------------------------
 \vspace{-0.4cm}

 \begin{longtable}{L{4cm}@{\hskip 0.15in}L{13.5cm}}
 \talkentry{April 2024} 
	{{FIREFLy-SOSS: Exoplanet Transit Light Curves Extraction Pipeline for JWST NIRISS-SOSS Observations}}
	{Departmental Undergraduate Research Showcase, Johns Hopkins University, MD}
 
         \talkentry{April 2024} 
	{{Is The Formation Of Planets The Cause of Solar Atypical Abundance Pattern?}}
	{Johns Hopkins University DREAMS Symposium}

        \talkentry{April 2024} 
	{{Characterization of Cloud-free Hot-Saturn WASP-96b with Joint JWST, Hubble, VLT, and Spitzer Transmission Spectroscopy}}
	{Johns Hopkins University DREAMS Symposium}
    
    \talkentry{Jan 2024}{\href{https://ui.adsabs.harvard.edu/abs/2024AAS...24310605W/abstract}{Elemental Abundance Trends with Condensation Temperature are Unrelated to Planet Formation}}{243rd Meeting of the American Astronomical Society, New Orleans, LA}
    
        \talkentry{June 2023} 
	{{Elemental Abundance Trends with Condensation Temperature are Unrelated to Planet Formation}}
	{Origins of Solar Systems Gordon Research Conference, Mount Holyoke College, MA}

     \talkentry{June 2023} 
	{{Stellar Elemental Abundance Patterns: Implications for Planet Formation}}
	{No-PhD Journal Club, Johns Hopkins University, MD}
\pagebreak
	\talkentry{Aug. 2022} 
	{\href{https://sites.krieger.jhu.edu/jhu-care/summer-2022/}{Optimizing JWST BOTS Transit Light Curve Fitting}}
	{The Center for Astrophysics Research Experience, Johns Hopkins University, MD}
\end{longtable}

\section{Telescope Allocations}
\vspace{-0.4cm}

 \begin{longtable}{L{4cm}@{\hskip 0.15in}L{13.5cm}}
 \telescope{2024 Q3}{Apache Point Observatory, ARCTIC, 3 nights}{Synergistic Cool Star Monitoring: Characterization of Starspots}{PIs: Rustamkulov, Z., Allen, N., \textbf{Wang, C. L.}, Wang, G.}
\end{longtable}

% Example \tableentry{} command to add another line:

%\tableentry{Heading}{Content}{spaceafter}

% All 3 parameters must be supplied but any can be empty if you don't need them
% A "spaceafter" value in the third parameter will add some vertical space -- this is to be used between headings

%------------------------------------------------

 % Start a table with two columns, the table will ensure everything is aligned
	
	%------------------------------------------------

%----------------------------------------------------------------------------------------
%	TEACHING
%----------------------------------------------------------------------------------------

\section{Teaching appointments} 
%------------------------------------------------
\vspace{-0.4cm}

    \begin{longtable}{L{4cm}@{\hskip 0.15in}L{13.5cm}}
    \teaching{2024 Spring}{Teaching Assistant, AS.171.108 General Physics II (Undergraduate, 23 students)}
    \teaching{2023 Fall}{Teaching Assistant, AS.171.107 General Physics I (Undergraduate, 46 students)}
    \teaching{2023 Spring}{Teaching Assistant, AS.171.101 General Physics I (Undergraduate, 46 students)}
    \teaching{2022 Fall}{Teaching Assistant, AS.171.101 General Physics I (Undergraduate, 23 students)}
    \end{longtable}

\section{Skills} 
%------------------------------------------------

% Blank \educationentry{} command to add another degree:

%\educationentry{} % Duration
%{} % Degree
%{} % Honours, achievements or distinctions (e.g. first class honours)
%{} % Department
%{} % Institution

% All 5 parameters must be supplied but any can be empty if you don't need them

%------------------------------------------------
\vspace{-0.4cm}


\begin{longtable}{L{4cm}@{\hskip 0.15in}L{13.5cm}} % Start a table with two columns, the table will ensure everything is aligned

	%------------------------------------------------
	
	
	%------------------------------------------------
	
	\skill{Computer Slangs} % Duration
	{Python, C/C++, Java, Assembly, Fortran, Matlab, R, HTML, CSS, JavaScript, Bash} % 
	%------------------------------------------------
	
	\skill{Languages} % Duration
	{English, Chinese, French} % Degree
	%------------------------------------------------
	
	\skill{Astronomy Softwares} % Duration
	{DS9, Siril, MESA (stellar structure), Rebound (N-body), Mercury (N-body), petitRADTRANS (atmospheric retrieval)} % 
	
	%------------------------------------------------

 	\skill{Observation Experience} % Duration
	{The Morris W. Offit Telescope (half-meter telescope at JHU), ARC 3.5m telescope at Apache Point Observatory} % Degree

	%------------------------------------------------
	
	
	\skill{Other} % Duration
	{Pytorch, \LaTeX, Git, Slurm, Mathematica, JupyterLab, Adobe Lightroom, Adobe Photoshop, Blender, Soccer, A Cappella, Marathon} 
\end{longtable}

\end{document}